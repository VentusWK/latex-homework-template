\documentclass{article}

\usepackage{fancyhdr}
\usepackage{extramarks}
\usepackage{amsmath}
\usepackage{amsthm}
\usepackage{amsfonts}
\usepackage{tikz}
\usepackage[plain]{algorithm}
\usepackage{algpseudocode}
\usepackage{graphicx}
\graphicspath{{./images/}}

\usetikzlibrary{automata,positioning}

%
% Basic Document Settings
%

\topmargin=-0.45in
\evensidemargin=0in
\oddsidemargin=0in
\textwidth=6.5in
\textheight=9.0in
\headsep=0.25in

\linespread{1.1}

\pagestyle{fancy}
\lhead{\hmwkAuthorName}
\chead{\hmwkClass\ (\hmwkClassInstructor\ \hmwkClassTime): \hmwkTitle}
\rhead{\firstxmark}
\lfoot{\lastxmark}
\cfoot{\thepage}

\renewcommand\headrulewidth{0.4pt}
\renewcommand\footrulewidth{0.4pt}

\setlength\parindent{0pt}

%
% Create Problem Sections
%

\newcommand{\enterProblemHeader}[1]{
    \nobreak\extramarks{}{Problem \arabic{#1} continued on next page\ldots}\nobreak{}
    \nobreak\extramarks{Problem \arabic{#1} (continued)}{Problem \arabic{#1} continued on next page\ldots}\nobreak{}
}

\newcommand{\exitProblemHeader}[1]{
    \nobreak\extramarks{Problem \arabic{#1} (continued)}{Problem \arabic{#1} continued on next page\ldots}\nobreak{}
    \stepcounter{#1}
    \nobreak\extramarks{Problem \arabic{#1}}{}\nobreak{}
}

\setcounter{secnumdepth}{0}
\newcounter{partCounter}
\newcounter{homeworkProblemCounter}
\setcounter{homeworkProblemCounter}{1}
\nobreak\extramarks{Problem \arabic{homeworkProblemCounter}}{}\nobreak{}

%
% Homework Problem Environment
%
% This environment takes an optional argument. When given, it will adjust the
% problem counter. This is useful for when the problems given for your
% assignment aren't sequential. See the last 3 problems of this template for an
% example.
%
\newenvironment{homeworkProblem}[1][-1]{
    \ifnum#1>0
        \setcounter{homeworkProblemCounter}{#1}
    \fi
    \section{Problem \arabic{homeworkProblemCounter}}
    \setcounter{partCounter}{1}
    \enterProblemHeader{homeworkProblemCounter}
}{
    \exitProblemHeader{homeworkProblemCounter}
}

%
% Homework Details
%   - Title
%   - Due date
%   - Class
%   - Section/Time
%   - Instructor
%   - Author
%

\newcommand{\hmwkTitle}{Assignment\ \#8}
\newcommand{\hmwkDueDate}{November 11, 2022}
\newcommand{\hmwkClass}{Formal}
\newcommand{\hmwkClassTime}{4:10 PM}
\newcommand{\hmwkClassInstructor}{Professor Matthew Patitz}
\newcommand{\hmwkAuthorName}{\textbf{Piam Chittisane}}

%
% Title Page
%

\title{
    \vspace{2in}
    \textmd{\textbf{\hmwkClass:\ \hmwkTitle}}\\
    \normalsize\vspace{0.1in}\small{Due\ on\ \hmwkDueDate\ at 4:00 PM}\\
    \vspace{0.1in}\large{\textit{\hmwkClassInstructor\ \hmwkClassTime}}
    \vspace{3in}
}

\author{\hmwkAuthorName}
\date{}

\renewcommand{\part}[1]{\textbf{\large Part \Alph{partCounter}}\stepcounter{partCounter}\\}

%
% Various Helper Commands
%

% Useful for algorithms
\newcommand{\alg}[1]{\textsc{\bfseries \footnotesize #1}}

% For derivatives
\newcommand{\deriv}[1]{\frac{\mathrm{d}}{\mathrm{d}x} (#1)}

% For partial derivatives
\newcommand{\pderiv}[2]{\frac{\partial}{\partial #1} (#2)}

% Integral dx
\newcommand{\dx}{\mathrm{d}x}

% Alias for the Solution section header
\newcommand{\solution}{\textbf{\large Solution}}

% Probability commands: Expectation, Variance, Covariance, Bias
\newcommand{\E}{\mathrm{E}}
\newcommand{\Var}{\mathrm{Var}}
\newcommand{\Cov}{\mathrm{Cov}}
\newcommand{\Bias}{\mathrm{Bias}}

\begin{document}

\maketitle

\pagebreak

\begin{homeworkProblem}
    Create a turing machine T such that for an input A where A is a DFA:
    \begin{enumerate}
        \item If every state in A is an accept state, accept. Otherwise, reject.
    \end{enumerate}
\end{homeworkProblem}

\begin{homeworkProblem}
    Proof by contradiction \\
    Suppose a bijection f exists between B and N. \\
    We can create a sequence $x \in B$ such that x isn't paired with anything in N. \\
    If we take the $n$th digit of the $n$th sequence of B and change it to anything else (arbitrarily), we can create a new $x$ on the diagonal that is in B, but doesn't correspond to any of the natural numbers.
    This is a contradiction, meaning that B must be uncountable.
\end{homeworkProblem}

\begin{homeworkProblem}
    We can map the three tuple $(i, j, k)$ to the infinite subset of N where $f(n) = 2^i3^j5^k$. Since the prime factorizations of a number is unique to itself, we can guarantee the relationship is one-to-one. The relationship is also onto since we can generate any prime factorization of the form $2^i3^j5^k$ with any arbitrary i, j, or k. This means a bijection exists, meaning the set T is countable. 
\end{homeworkProblem}

\begin{homeworkProblem}
    Construct a turing machine T that takes in the encoding of a DFA M and string w:
    \begin{enumerate}
        \item Construct a string $w^R$
        \item Run string w through DFA M
        \item If $w$ is accepted by M, run string $w^R$ through DFA M, otherwise T rejects
        \item If $w^R$ is accepted by M, T accepts, otherwise T rejects
    \end{enumerate}
    S is decidable.
\end{homeworkProblem}

\begin{homeworkProblem}
    Proof by contradiction \\
    Assume T is a decidable language and that R decides it.
    Construct a turing machine J such that on input $\langle M, w\rangle$ where M is a turing machine and w is the input string:
    \begin{enumerate}
        \item Create a turing machine S such that on an input $x$
        \begin{enumerate}
            \item if x has the form 01, accept
            \item else, run the input w on turing machine M and accept if M accepts w, otherwise reject
        \end{enumerate}
        \item run R on $\langle S\rangle$
        \item If R accepts, J accepts, reject otherwise.
    \end{enumerate}
\end{homeworkProblem}
\end{document}
